\documentclass{article}

% Pakiety
\usepackage[utf8]{inputenc}
\usepackage[T1]{fontenc}
% \usepackage[polish]{babel}
\usepackage{graphicx}
\usepackage{hyperref}
\usepackage{listings}
\usepackage{color}
\usepackage{float}
\usepackage{geometry}
\usepackage{tikz}
\usepackage{fancyhdr}
\usepackage{titlesec}

% Konfiguracja strony
\geometry{a4paper, margin=1.5cm}

% Konfiguracja nagłówków i stopek
\pagestyle{fancy}
\fancyhf{}
\fancyhead[L]{System monitorowania produkcji przemysłowej IoT}
\fancyhead[R]{\thepage}
\renewcommand{\headrulewidth}{0.4pt}
\renewcommand{\footrulewidth}{0.4pt}

% Style sekcji
\titleformat{\section}{\normalfont\Large\bfseries}{\thesection}{1em}{}
\titleformat{\subsection}{\normalfont\large\bfseries}{\thesubsection}{1em}{}

% Konfiguracja kolorów dla listingów
\definecolor{codegreen}{rgb}{0,0.6,0}
\definecolor{codegray}{rgb}{0.5,0.5,0.5}
\definecolor{codepurple}{rgb}{0.58,0,0.82}
\definecolor{backcolour}{rgb}{0.95,0.95,0.92}

% Definicja języka C# dla listingów
\lstdefinelanguage{CSharp}{
  morestring=[b]",
  morestring=[b]',
  morecomment=[l]{//},
  morecomment=[s]{/*}{*/},
  morekeywords={abstract,event,new,struct,as,explicit,null,switch,base,extern,object,this,bool,false,operator,throw,break,finally,out,true,byte,fixed,override,try,case,float,params,typeof,catch,for,private,uint,char,foreach,protected,ulong,checked,goto,public,unchecked,class,if,readonly,unsafe,const,implicit,ref,ushort,continue,in,return,using,decimal,int,sbyte,virtual,default,interface,sealed,volatile,delegate,internal,short,void,do,is,sizeof,while,double,lock,stackalloc,else,long,static,enum,namespace,string,var,dynamic,async,await,get,set,where,yield,partial,add,remove}
}

% Definicja języka JSON dla listingów
\lstdefinelanguage{JSON}{
  morestring=[b]",
  morekeywords={true,false,null},
  sensitive=true,
  comment=[l]{//}
}

% Style dla C#
\lstset{
  extendedchars=true,
  inputencoding=utf8
}

\lstdefinestyle{csharp}{
    backgroundcolor=\color{backcolour},
    commentstyle=\color{codegreen},
    keywordstyle=\color{codepurple},
    stringstyle=\color{codegreen},
    basicstyle=\small\ttfamily,
    breakatwhitespace=false,
    breaklines=true,
    captionpos=b,
    keepspaces=true,
    numbers=left,
    numbersep=5pt,
    showspaces=false,
    showstringspaces=false,
    showtabs=false,
    tabsize=2,
    language=CSharp
}

% Style dla JSON
\lstdefinestyle{json}{
    backgroundcolor=\color{backcolour},
    commentstyle=\color{codegreen},
    keywordstyle=\color{blue},
    stringstyle=\color{codegreen},
    basicstyle=\small\ttfamily,
    breakatwhitespace=false,
    breaklines=true,
    captionpos=b,
    keepspaces=true,
    numbers=left,
    numbersep=5pt,
    showspaces=false,
    showstringspaces=false,
    showtabs=false,
    tabsize=2,
    language=JSON
}

\begin{document}

\begin{titlepage}
    \centering
    \vspace*{1cm}
    {\Huge\bfseries System monitorowania produkcji przemysłowej IoT \par}
    \vspace{1.5cm}
    {\Large Dokumentacja projektowa \par}
    \vspace{2cm}
    {\LARGE\textit{Autor:} \par}
    \vspace{0.5cm}
    {\large Bartosz Gruda \par}
    \vfill
    {\large \today \par}
\end{titlepage}

\tableofcontents
\newpage

\section{Wprowadzenie}
\subsection{Cel projektu}
Celem projektu jest implementacja systemu monitorowania produkcji przemysłowej opartego na technologiach Internet of Things (IoT). System umożliwia zbieranie danych telemetrycznych z urządzeń przemysłowych za pomocą protokołu OPC UA, przetwarzanie ich w chmurze Microsoft Azure oraz analizę w czasie rzeczywistym z wykorzystaniem Azure Stream Analytics.

\subsection{Zakres projektu}
Projekt obejmuje następujące komponenty:
\begin{itemize}
    \item Agent IoT (\texttt{IndustrialIoT.Agent}) - aplikacja .NET Core komunikująca się z urządzeniami OPC UA
    \item Funkcje Azure (\texttt{IndustrialIoT.Functions}) - logika biznesowa przetwarzania danych w chmurze
    \item Zapytania Stream Analytics - przetwarzanie danych w czasie rzeczywistym
    \item Integracja z Azure IoT Hub jako bramą do chmury
\end{itemize}

\section{Architektura systemu}
\subsection{Schemat architektury}

System składa się z następujących komponentów połączonych w poniższy sposób:

\begin{center}
    \begin{tikzpicture}[node distance=2cm]
        % Węzły
        \node (devices) [rectangle, draw, minimum width=3cm, minimum height=1cm] {Urządzenia OPC UA};
        \node (agent) [rectangle, draw, minimum width=3cm, minimum height=1cm, right of=devices, xshift=2.5cm] {IoT Agent (.NET Core)};
        \node (hub) [rectangle, draw, minimum width=3cm, minimum height=1cm, right of=agent, xshift=2.5cm] {Azure IoT Hub};
        \node (stream) [rectangle, draw, minimum width=3cm, minimum height=1cm, right of=hub, xshift=2.5cm] {Stream Analytics};
        \node (functions) [rectangle, draw, minimum width=3cm, minimum height=1cm, below of=stream, yshift=-1cm] {Azure Functions};

        % Połączenia
        \draw[->, thick] (devices) -- (agent);
        \draw[->, thick] (agent) -- (hub);
        \draw[->, thick] (hub) -- (stream);
        \draw[->, thick] (stream) -- (functions);
    \end{tikzpicture}
\end{center}

\subsection{Opis komponentów}
\subsubsection{Urządzenia OPC UA}
System obsługuje dowolne urządzenia przemysłowe zgodne ze standardem OPC UA. Urządzenia te mogą być fizycznymi sterownikami PLC, czujnikami, lub symulatorami OPC UA używanymi w środowisku testowym.

\subsubsection{IndustrialIoT.Agent}
Agent IoT to aplikacja .NET Core, która:
\begin{itemize}
    \item Nawiązuje połączenia z urządzeniami OPC UA
    \item Zbiera dane telemetryczne (temperatura, liczniki produkcji, błędy)
    \item Przesyła dane do Azure IoT Hub w bezpieczny sposób
    \item Obsługuje konfigurację z pliku \texttt{AppSettings.json}
\end{itemize}

\subsubsection{Azure IoT Hub}
Azure IoT Hub pełni rolę centralnej bramy zarządzającej komunikacją:
\begin{itemize}
    \item Obsługuje komunikację dwukierunkową między urządzeniami a chmurą
    \item Zapewnia bezpieczną komunikację za pomocą protokołu MQTT/HTTPS
    \item Zarządza tożsamościami urządzeń
    \item Umożliwia routing wiadomości do innych serwisów Azure
\end{itemize}

\subsubsection{Stream Analytics}
Azure Stream Analytics przetwarza dane w czasie rzeczywistym:
\begin{itemize}
    \item Oblicza kluczowe wskaźniki wydajności (KPI) produkcji
    \item Monitoruje statystyki temperatury
    \item Wykrywa błędy urządzeń i alarmuje o anomaliach
\end{itemize}

\subsubsection{IndustrialIoT.Functions}
Azure Functions implementują logikę biznesową:
\begin{itemize}
    \item Przetwarzają wiadomości z IoT Hub
    \item Implementują analizy danych
    \item Udostępniają API do pobierania zagregowanych informacji
\end{itemize}

\section{Implementacja komponentów}
\subsection{IndustrialIoT.Agent}
\subsubsection{Struktura projektu}
Projekt agenta składa się z następujących kluczowych plików:
\begin{itemize}
    \item \texttt{Program.cs} - punkt wejścia aplikacji
    \item \texttt{IoTDevice.cs} - klasa bazowa dla urządzeń IoT
    \item \texttt{OpcDevice.cs} - implementacja urządzenia OPC UA
    \item \texttt{AppSettings.json} - plik konfiguracyjny
\end{itemize}

\subsubsection{Klasa IoTDevice}
Klasa \texttt{IoTDevice} jest klasą bazową dla wszystkich urządzeń IoT w systemie. Implementuje podstawową funkcjonalność:

\begin{lstlisting}[style=csharp, caption=IoTDevice.cs (fragment)]
public abstract class IoTDevice
{
    protected string DeviceId { get; set; }
    protected string ConnectionString { get; set; }
    protected DeviceClient DeviceClient { get; set; }

    public IoTDevice(string deviceId, string connectionString)
    {
        DeviceId = deviceId;
        ConnectionString = connectionString;
        DeviceClient = DeviceClient.CreateFromConnectionString(ConnectionString);
    }

    public abstract Task Initialize();
    public abstract Task SendTelemetryAsync();
    public abstract Task ReceiveCommands();
}
\end{lstlisting}

\subsubsection{Klasa OpcDevice}
Klasa \texttt{OpcDevice} rozszerza klasę \texttt{IoTDevice} i implementuje komunikację z urządzeniami OPC UA:

\begin{lstlisting}[style=csharp, caption=OpcDevice.cs (fragment)]
public class OpcDevice : IoTDevice
{
    private OpcClient Client { get; set; }
    private string EndpointUrl { get; set; }
    private string NodeId { get; set; }

    public OpcDevice(string deviceId, string connectionString, 
                     string endpointUrl, string nodeId)
        : base(deviceId, connectionString)
    {
        EndpointUrl = endpointUrl;
        NodeId = nodeId;
    }

    public override async Task Initialize()
    {
        Client = new OpcClient(EndpointUrl);
        await Client.Connect();
        // Inicjalizacja urzadzenia OPC
    }

    public override async Task SendTelemetryAsync()
    {
        // Odczyt danych z OPC UA
        var temperatureValue = await Client.ReadNodeAsync(NodeId + "/Temperature");
        var goodCountValue = await Client.ReadNodeAsync(NodeId + "/GoodCount");
        var badCountValue = await Client.ReadNodeAsync(NodeId + "/BadCount");
        var statusValue = await Client.ReadNodeAsync(NodeId + "/Status");
        
        // Przygotowanie telemetrii
        var telemetryData = new
        {
            temperature = (double)temperatureValue,
            goodCount = (int)goodCountValue,
            badCount = (int)badCountValue,
            productionStatus = statusValue.ToString(),
            deviceErrors = GetDeviceErrors()
        };
        
        // Wyslanie telemetrii do IoT Hub
        var messageString = JsonConvert.SerializeObject(telemetryData);
        var message = new Message(Encoding.UTF8.GetBytes(messageString));
        await DeviceClient.SendEventAsync(message);
    }
    
    // Pozostale metody implementacyjne
}
\end{lstlisting}

\subsection{IndustrialIoT.Functions}
\subsubsection{Struktura projektu}
Projekt Azure Functions zawiera następujące kluczowe pliki:
\begin{itemize}
    \item \texttt{Models.cs} - modele danych dla przetwarzania wiadomości IoT
    \item \texttt{AzureIoTService.cs} - usługa integracji z Azure IoT Hub
    \item \texttt{BusinessLogicFunctions.cs} - funkcje zawierające logikę biznesową
    \item \texttt{Startup.cs} - konfiguracja aplikacji
    \item \texttt{host.json} - konfiguracja hosta Functions
    \item \texttt{local.settings.json} - lokalne ustawienia dla środowiska deweloperskiego
\end{itemize}

\subsubsection{Modele danych}
Plik \texttt{Models.cs} definiuje struktury danych używane w systemie:

\begin{lstlisting}[style=csharp, caption=Models.cs (fragment)]
namespace IndustrialIoT.Functions
{
    public class TelemetryMessage
    {
        public string DeviceId { get; set; }
        public DateTime Timestamp { get; set; }
        public double Temperature { get; set; }
        public int GoodCount { get; set; }
        public int BadCount { get; set; }
        public string ProductionStatus { get; set; }
        public string DeviceErrors { get; set; }
    }
    
    public class ProductionKpi
    {
        public string DeviceId { get; set; }
        public double QualityPercentage { get; set; }
        public int TotalGood { get; set; }
        public int TotalBad { get; set; }
        public DateTime WindowEnd { get; set; }
    }
    
    public class TemperatureStats
    {
        public string DeviceId { get; set; }
        public double AverageTemperature { get; set; }
        public double MinTemperature { get; set; }
        public double MaxTemperature { get; set; }
        public DateTime WindowEnd { get; set; }
    }
    
    public class DeviceError
    {
        public string DeviceId { get; set; }
        public int ErrorCount { get; set; }
        public DateTime WindowEnd { get; set; }
    }
}
\end{lstlisting}

\subsubsection{Integracja z Azure IoT Hub}
Plik \texttt{AzureIoTService.cs} zawiera implementację usługi do komunikacji z Azure IoT Hub:

\begin{lstlisting}[style=csharp, caption=AzureIoTService.cs (fragment)]
public class AzureIoTService
{
    private readonly ServiceClient _serviceClient;
    private readonly RegistryManager _registryManager;
    
    public AzureIoTService(string connectionString)
    {
        _serviceClient = ServiceClient.CreateFromConnectionString(connectionString);
        _registryManager = RegistryManager.CreateFromConnectionString(connectionString);
    }
    
    public async Task<Device> GetDeviceAsync(string deviceId)
    {
        return await _registryManager.GetDeviceAsync(deviceId);
    }
    
    public async Task SendCommandAsync(string deviceId, string commandName, object payload)
    {
        var commandMessage = new Message(Encoding.UTF8.GetBytes(JsonConvert.SerializeObject(payload)));
        await _serviceClient.SendAsync(deviceId, commandMessage);
    }
    
    // Pozostale metody do zarzadzania urzadzeniami i wiadomosciami
}
\end{lstlisting}

\subsubsection{Funkcje biznesowe}
Plik \texttt{BusinessLogicFunctions.cs} zawiera implementację funkcji Azure obsługujących logikę biznesową:

\begin{lstlisting}[style=csharp, caption=BusinessLogicFunctions.cs (fragment)]
public class BusinessLogicFunctions
{
    private readonly AzureIoTService _ioTService;
    
    public BusinessLogicFunctions(AzureIoTService ioTService)
    {
        _ioTService = ioTService;
    }
    
    [FunctionName("ProcessTelemetry")]
    public async Task ProcessTelemetry(
        [IoTHubTrigger("messages/events", Connection = "IoTHubConnection")]
        EventData message,
        ILogger log)
    {
        try
        {
            string messageBody = Encoding.UTF8.GetString(message.Body.Array);
            var telemetry = JsonConvert.DeserializeObject<TelemetryMessage>(messageBody);
            
            log.LogInformation($"Otrzymano telemetrie z urzadzenia: {telemetry.DeviceId}");
            
            // Przetwarzanie danych telemetrycznych
            // ...
            
            // Zapisanie danych w bazie danych
            // ...
        }
        catch (Exception ex)
        {
            log.LogError($"Blad przetwarzania telemetrii: {ex.Message}");
        }
    }
    
    [FunctionName("GetDeviceStatus")]
    public async Task<IActionResult> GetDeviceStatus(
        [HttpTrigger(AuthorizationLevel.Function, "get", Route = "devices/{deviceId}/status")]
        HttpRequest req,
        string deviceId,
        ILogger log)
    {
        try
        {
            var device = await _ioTService.GetDeviceAsync(deviceId);
            if (device == null)
            {
                return new NotFoundResult();
            }
            
            // Pobranie statusu urzadzenia
            // ...
            
            return new OkObjectResult(deviceStatus);
        }
        catch (Exception ex)
        {
            log.LogError($"Blad pobierania statusu urzadzenia: {ex.Message}");
            return new StatusCodeResult(StatusCodes.Status500InternalServerError);
        }
    }
    
    // Pozostale funkcje API i obslugi zdarzen
}
\end{lstlisting}

\subsection{Stream Analytics}
\subsubsection{Zapytania}
Plik \texttt{stream\_analitics\_kwerendy.txt} zawiera zapytania używane w Azure Stream Analytics do przetwarzania danych w czasie rzeczywistym:

\begin{lstlisting}[caption=Zapytanie obliczające KPI produkcji]
-- Production KPIs
SELECT
    deviceId,
    CASE
        WHEN SUM(CAST(goodCount AS float)) + SUM(CAST(badCount AS float)) > 0
        THEN (SUM(CAST(goodCount AS float)) * 100.0 / (SUM(CAST(goodCount AS float)) + SUM(CAST(badCount AS float))))
        ELSE 100.0
    END AS prod_kpi,
    SUM(CAST(goodCount AS float)) AS total_good,
    SUM(CAST(badCount AS float)) AS total_bad,
    System.Timestamp() AS windowEnd
INTO
    kpi-output
FROM
    "iot-hub"
WHERE
    productionStatus = 'Running' OR productionStatus = '1'
GROUP BY
    deviceId,
    TumblingWindow(minute, 5)
HAVING
    SUM(CAST(goodCount AS float)) + SUM(CAST(badCount AS float)) > 0;
\end{lstlisting}

\begin{lstlisting}[caption=Zapytanie monitorujące statystyki temperatury]
-- Temperature Stats
SELECT
    deviceId,
    AVG(temperature) AS avg_t,
    MIN(temperature) AS min_t,
    MAX(temperature) AS max_t,
    System.Timestamp() AS windowEnd
INTO
    temperature-stats
FROM
    "iot-hub"
GROUP BY
    deviceId,
    SlidingWindow(minute, 5)
HAVING
    COUNT(*) > 0;
\end{lstlisting}

\begin{lstlisting}[caption=Zapytanie wykrywające błędy urządzeń]
-- Device Errors
SELECT
    deviceId,
    COUNT(*) AS errorCount,
    System.Timestamp() AS windowEnd
INTO
    error-data
FROM
    "iot-hub"
WHERE
    deviceErrors IS NOT NULL AND deviceErrors <> '0' AND deviceErrors <> 'None'
    OR (errorCode IS NOT NULL AND errorCode > 0)
GROUP BY
    deviceId,
    TumblingWindow(minute, 1)
HAVING
    COUNT(*) > 3;
\end{lstlisting}

\section{Bezpieczeństwo systemu}
\subsection{Mechanizmy uwierzytelniania}
System wykorzystuje następujące mechanizmy bezpieczeństwa do uwierzytelniania urządzeń i komunikacji:
\begin{itemize}
    \item Sygnatury SAS (Shared Access Signatures) do bezpiecznego uwierzytelniania urządzeń w Azure IoT Hub
    \item Certyfikaty X.509 dla komunikacji z urządzeniami OPC UA
    \item Azure Active Directory do uwierzytelniania użytkowników w aplikacjach zaplecza
\end{itemize}

\subsection{Zabezpieczenie transmisji danych}
Komunikacja między komponentami systemu jest zabezpieczona następującymi metodami:
\begin{itemize}
    \item Szyfrowanie TLS 1.2 lub nowsze dla wszystkich połączeń sieciowych
    \item Bezpieczne tunele komunikacyjne dla urządzeń za zaporami sieciowymi
    \item Szyfrowanie przechowywanych danych w Azure Storage i Cosmos DB
\end{itemize}

\subsection{Zarządzanie tożsamościami urządzeń}
Azure IoT Hub zapewnia:
\begin{itemize}
    \item Unikalną tożsamość dla każdego urządzenia
    \item Możliwość zdalnego wyłączenia skompromitowanych urządzeń
    \item Rotację kluczy dostępowych dla zwiększenia bezpieczeństwa
\end{itemize}

\section{Konfiguracja i wdrożenie}
\subsection{Wymagania wstępne}
Do uruchomienia systemu potrzebne są:
\begin{itemize}
    \item .NET 6.0 SDK lub nowszy
    \item Subskrypcja Azure
    \item Instancja Azure IoT Hub
    \item Zadanie Stream Analytics
    \item Urządzenia zgodne z OPC UA lub symulatory
\end{itemize}

\subsection{Konfiguracja lokalnego środowiska}
\subsubsection{Konfiguracja agenta IoT}
Aby skonfigurować agenta IoT, należy zaktualizować plik \texttt{AppSettings.json}:

\begin{lstlisting}[style=json, caption=AppSettings.json]
{
  "IoTHubConnectionString": "HostName=your-iot-hub.azure-devices.net;DeviceId=your-device-id;SharedAccessKey=your-access-key",
  "DeviceId": "device1",
  "OpcUaEndpoint": "opc.tcp://localhost:4840",
  "OpcUaNodeId": "ns=2;s=Machine1",
  "TelemetryInterval": 1000,
  "LogLevel": "Information"
}
\end{lstlisting}

\subsubsection{Konfiguracja Azure Functions}
Aby skonfigurować Azure Functions, należy zaktualizować plik \texttt{local.settings.json}:

\begin{lstlisting}[style=json, caption=local.settings.json]
{
  "IsEncrypted": false,
  "Values": {
    "AzureWebJobsStorage": "UseDevelopmentStorage=true",
    "FUNCTIONS_WORKER_RUNTIME": "dotnet",
    "IoTHubConnection": "HostName=your-iot-hub.azure-devices.net;SharedAccessKeyName=service;SharedAccessKey=your-access-key",
    "CosmosDBConnection": "AccountEndpoint=https://your-account.documents.azure.com:443/;AccountKey=your-account-key;"
  }
}
\end{lstlisting}

\subsection{Uruchomienie lokalne}
\subsubsection{Uruchomienie agenta IoT}
Aby uruchomić agenta IoT lokalnie:
\begin{lstlisting}[language=bash]
cd IndustrialIoT.Agent
dotnet run
\end{lstlisting}

\subsubsection{Uruchomienie Azure Functions}
Aby uruchomić Azure Functions lokalnie:
\begin{lstlisting}[language=bash]
cd IndustrialIoT.Functions
func start
\end{lstlisting}

\subsection{Wdrożenie na Azure}
\subsubsection{Wdrożenie Azure Functions}
Aby wdrożyć Azure Functions:
\begin{lstlisting}[language=bash]
cd IndustrialIoT.Functions
func azure functionapp publish <nazwa-aplikacji-funkcji>
\end{lstlisting}

\subsubsection{Konfiguracja Stream Analytics}
Aby skonfigurować Stream Analytics:
\begin{enumerate}
    \item Utwórz nowe zadanie Stream Analytics w portalu Azure
    \item Skonfiguruj wejście ze źródła IoT Hub
    \item Skonfiguruj wyjścia dla danych KPI, statystyk temperatury i błędów urządzeń
    \item Dodaj zapytania z pliku \texttt{stream\_analitics\_kwerendy.txt}
    \item Uruchom zadanie
\end{enumerate}

\subsubsection{Wdrożenie agenta IoT}
Aby wdrożyć agenta IoT na urządzenie brzegowe:
\begin{enumerate}
    \item Opublikuj aplikację jako samodzielny plik wykonywalny
    \item Skopiuj pliki na urządzenie brzegowe
    \item Zaktualizuj \texttt{AppSettings.json} odpowiednimi wartościami konfiguracyjnymi
    \item Uruchom aplikację
\end{enumerate}

\section{Bibliografia}
\begin{enumerate}
    \item Microsoft Azure IoT Hub Documentation, \url{https://docs.microsoft.com/en-us/azure/iot-hub/}
    \item OPC Foundation, OPC UA Specification, \url{https://opcfoundation.org/about/opc-technologies/opc-ua/}
    \item Microsoft Azure Stream Analytics Documentation, \url{https://docs.microsoft.com/en-us/azure/stream-analytics/}
    \item Microsoft Azure Functions Documentation, \url{https://docs.microsoft.com/en-us/azure/azure-functions/}
\end{enumerate}

\end{document}